\chapter{Intelligenztest}
\label{iq}
\medskip
Datum: \rule{3cm}{0,4pt}
\bigskip

Lesen Sie bitte zuerst alle Fragen gründlich durch, bevor Sie sie beantworten. Sie haben insgesamt drei Minuten Zeit.

\begin{enumerate}
\item Wer komponierte die Oper \emph{Aida?} \hrulefill
\item Wer schrieb das Buch \emph{Krieg und Frieden?} \hrulefill
\item Wo fand 1954 die Fußballweltmeisterschaft statt? \hrulefill
\item Setzen Sie die Zahlenreihe fort: 2~--~4~--~6~-- \hrulefill
\item Von wem stammen die Figuren \emph{Max und Moritz?} \hrulefill
\item Wann lebte Karl der Große? \hrulefill
\item Wie viele Kontinente gibt es auf der Erde? \hrulefill
\item Wer erfand die Glühbirne? \hrulefill
\item Welches Land produziert das meiste Öl? \hrulefill
\item Wie heißt das westliche Verteidigungsbündnis? \hrulefill
\item Wie viele Tierkreiszeichen gibt es (Stier, Wassermann etc.)? Nur die Anzahl: \hrulefill
\item Durch wen wurde Napoleon besiegt? \hrulefill
\item Füllen Sie nur das heutige Datum oben links aus. Alles andere können Sie sich sparen. Genießen Sie noch zwei Minuten Ruhe.
\end{enumerate}

\chapter{Kennenlern-Bingo}
\label{bingo}
Gehe im Raum herum und finde Personen, die den Anforderungen in den Kästchen entsprechen. Für jedes Kästchen soll eine Person gefunden werden, die dann in dem entsprechenden Kästchen unterschreibt.

Wer vier Kästchen in einer Reihe ausgefüllt bekommt~-- waagerecht, senkrecht oder schräg~--, ruft laut \fett{"`Bingo!"'}. Je mehr \emph{Bingo}s, desto besser.

Denk daran, dass es Ziel des Spieles ist, die anderen TeilnehmerInnen kennen zu lernen: Deshalb unterhalte dich ruhig ein wenig mit ihnen, auch wenn du schon die Unterschrift hast!

\subsection*{Finde jemanden, der/die \ldots}

\renewcommand{\arraystretch}{1.27}
\noindent\begin{tabular}{|p{9.0em}|p{9.0em}|p{9.0em}|p{9.0em}|}
\hline
\ldots\ eine Sprache spricht, die du überhaupt nicht sprichst:&
\ldots\ ein Haustier hat:&
\ldots\ die gleiche Augenfarbe wie du hat:\vspace{5.5em}&
\ldots\ in den letzten drei Jahren einmal (als PatientIn) im Krankenhaus war:\\
\hline
\ldots\ mindestens ein Jahr außerhalb Deutschlands gelebt hat:&
\ldots\ das gleiche Musikinstrument spielt wie du:\vspace{4.5em}&
\ldots\ eine andere Staatsangehörigkeit als du hat:&
\ldots\ im gleichen Monat Geburtstag hat wie du:\\
\hline
\ldots\ den gleichen Lieblingsfilm hat wie du:&
\ldots\ etwas Handgemachtes trägt:\vspace{5.5em}&
\ldots\ im gleichen Jahr geboren wurde wie du:&
\ldots\ keinen Fernseher hat:\\
\hline
\ldots\ gleich viele Geschwister wie du hat:\vspace{5.5em}&
\ldots\ das gleiche Hauptverkehrsmittel wie du hat:&
\ldots\ in einem Land war, in dem du noch nie warst:&
\ldots\ denselben Sport wie du betreibt:\\
\hline
\end{tabular}
\renewcommand{\arraystretch}{1.0}

\chapter{NASA-Spiel}
\label{nasa-kopien}

Sie sind Mitglied einer Raumfahrtgruppe, die ursprünglich geplant hatte, auf der erhellten Oberfläche des Mondes mit einem Mutterschiff zusammenzutreffen. Infolge technischer Schwierigkeiten ist Ihr Raumschiff jedoch gezwungen worden, an einer Stelle in der Tagzone zu landen, die etwa 300~km von dem Treffpunkt entfernt liegt. Während der Landung ist ein großer Teil der Ausrüstung an Bord beschädigt worden.

Da die Aussicht zu überleben davon abhängt, ob Sie das Mutterschiff erreichen, müssen die wichtigsten der vorhandenen Dinge für den 300~km langen Weg gewählt werden. Unten finden Sie eine Liste von 15 Gegenständen, die nach der Landung unbeschädigt geblieben sind. Ihre Aufgabe ist es, diese Gegenstände in eine Rangordnung zu bringen, je nachdem, wie notwendig Sie Ihnen zum Erreichen des Treffpunktes erscheinen. Setzen Sie Nummer~1 neben den wichtigsten Gegenstand, Nummer~2 neben den zweitwichtigsten usw.

\subsection*{Liste der unbeschädigten Dinge}
\renewcommand{\arraystretch}{1.0}
\begin{tabular}{|lp{20em}|l|l|l|}
  \hline
  & & \multicolumn{3}{|c|}{\fett{Rangordnung}} \\
  \multicolumn{2}{|c|}{\fett{Artikel}} & \fett{Individuell} & \fett{Gruppe} & \fett{Plenum} \\
  \hline \hline
  1 & Schachtel Streichhölzer & & & \\
  \hline
  1 & Dose Nahrungskonzentrat pro Person (lässt sich mit einem Spezialventil an den Raumanzug anschließen) & & & \\
  \hline
  15~m & Nylonseil & & & \\
  \hline
  30~m$^2$ & Fallschirmseide (15\,$\times$\,2~m) & & & \\
  \hline
  1 & tragbares Heizgerät (mit Infrarot-Strahler als Wärmequelle) & & & \\
  \hline
  2 & Pistolen 7,654~mm & & & \\
  \hline
  1 & kleine Kiste Trockenmilch pro Person & & & \\
  \hline
  2 & Sauerstofftanks zu je 50~l pro Person & & & \\
  \hline
  1 & Sternenkarte (aus Mondperspektive) & & & \\
  \hline
  1 & Schlauchboot (automatisch aufblasbar durch integrierte CO$_2$-Kartuschen) & & & \\
  \hline
  1 & Magnetkompass & & & \\
  \hline
  22~l & Wasser pro Person (mit Spezialventil an den Raumanzug anschließbar) & & & \\
  \hline
  20 & Signalpatronen (auch im luftleeren Raum zündend, ohne Pistole abschießbar) & & & \\
  \hline
  1 & Erste-Hilfe-Koffer (u.\,a.~mit Injektionsnadeln) & & & \\
  \hline
  1 & Fernmelde-Empfänger und -Sender mit Solarzellen & & & \\
  \hline
\end{tabular}

\chapter{Schiffbruch}
\label{schiffbruch-kopien}
Sie segeln zu Weihnachten mit einer Privatjacht auf dem offenen Meer, etwa 800~Seemeilen südöstlich von Südafrika, als an Bord ein Brand ausbricht. Sie können nur noch die unten augelisteten 15~Gegenstände in das einzige vorhandene Rettungsboot mitnehmen. Da das Boot damit aber überlastet ist, müssen Sie sich einigen, welche Artikel Sie wegwerfen und welche Sie behalten wollen.

Ihre Aufgabe ist es, diese Gegenstände in eine Rangordnung zu bringen, je nachdem, wie notwendig Sie Ihnen in dieser Situation erscheinen. Setzen Sie Nummer 1 neben den wichtigsten Gegenstand, Nummer 2 neben den zweitwichtigsten usw.

Einige der Schiffbrüchigen haben außerdem Streichhölzer und Geld (Münzen und Scheine) dabei.

\subsection*{Die Liste der Gegenstände}

\renewcommand{\arraystretch}{1.27}
\begin{tabular}{|lp{20em}|l|l|l|}
  \hline
  & & \multicolumn{3}{|c|}{\fett{Rangordnung}} \\
  \multicolumn{2}{|c|}{\fett{Artikel}} & \fett{Individuell} & \fett{Gruppe} & \fett{Plenum} \\
  \hline \hline
  1 & Sextant (ohne weitere Dokumente) & & & \\
  \hline
  1 & Rasierspiegel & & & \\
  \hline
  25~l & Trinkwasser & & & \\
  \hline
  1 & großes Moskitonetz & & & \\
  \hline
  1 & Nahrungsration, reicht für 1~Tag (pro Person) & & & \\
  \hline
  1 & Karte des Indischen Ozeans & & & \\
  \hline
  1 & aufblasbares Kopfkissen & & & \\
  \hline
  24~l & Dieselöl & & & \\
  \hline
  1 & FM-Transistorradio & & & \\
  \hline
  1~l & Haifisch-Abwehr-Flüssigkeit & & & \\
  \hline
  10 m$^2$ & Plastikfolie & & & \\
  \hline
  1,5~l & Cognac & & & \\
  \hline
  5~m & Nylonschnur & & & \\
  \hline
  400~g & Schokolade (pro Person) & & & \\
  \hline
  1 & Angel mit Zubehör & & & \\
  \hline
\end{tabular}
\renewcommand{\arraystretch}{1.0}

\chapter{Insel ohne Wiederkehr}
\label{wiederkehr-kopien}
Nachdem Sie mit allen Personen und (erstaunlicherweise) allen Gegenständen von der Jacht fliehen, ohne dass das Rettungsboot sank, entdecken Sie noch einige weitere Gegenstände im Rettungsboot (fragen Sie besser nicht genauer nach, wie das Vieh dort hinein gekommen ist, warum es noch lebt und warum Sie es vorher nicht bemerkt haben \ldots).

\vspace*{.5em}
\begin{tabular}{rp{30em}}
  1 & Hammer \\
  1 & Säge \\
  1 & Packung Nägel \\
  1 & Gewehr \\
  500 & Schuss Munition für das Gewehr \\
  1 & Kuh \\
  1 & Stier \\
  1 & grobe Karte von Südafrika und dem südlich davon liegenden Meer (siehe Seite \pageref{wiederkehr-karte}) \\
\end{tabular}
\vspace*{.5em}

Auf der Karte ist zusätzlich noch eine Flug- und Schifffahrtslinie eingezeichnet, auf der in Ihnen nicht bekannten Abständen (Stunden? Wochen?) Flugzeuge und Schiffe verkehren. Mit dem verbleibenden Sprit wäre diese Linie gerade so zu erreichen. Wenn Sie dort von einem Flugzeug oder Schiff gesehen würden, könnten Sie gerettet werden und in die Zivilisation zurückkehren.

Weiterhin ist auf der Karte fernab aller Flug- und Schifffahrtslinien die \emph{Insel ohne Wiederkehr} eingezeichnet, auf die von allen Seiten Strömungen zulaufen. Mit dem verbleibenden Sprit wäre diese Insel ebenfalls gerade so zu erreichen. Wer einmal dort gestrandet ist, kann allerdings ohne Rettung von außen nicht mehr von dort entkommen. Die Überlebenschancen auf der Insel sind hingegen gut: Das Klima ist sehr angenehm, es gibt Wasser, viele essbare Pflanzen und nur wenig gefährliche Tiere.

Sie zeichnen auf der Karte sofort Ihre aktuelle Position ein. (Die Jacht ist inzwischen komplett gesunken.) Was tun Sie?

\pagebreak
\section*{Die Karte aus dem Rettungsboot}
\scalebox{0.4}[0.4]{\includegraphics*{insel-ohne-wiederkehr}}
\label{wiederkehr-karte}

\chapter{Trotzburg}
\section*{Auswertungsbogen}
\label{trotzburg-auswertung}
\enlargethispage{1,5cm}
\renewcommand{\arraystretch}{1.27}

\begin{tabular}{|l|l|l|}
  \hline
  \fett{Beobachtete Spielerin:} & \fett{Verteidigt sich:} & \fett{Gibt eigene Fehler zu:} \\
                                   \cline{2-3}
                                 & sachlich                & sachlich \\
                                   \cline{2-3}
                                 & unsachlich              & unsachlich \\

  \hline \hline

  \fett{Andere Spielerinnen:}    & \fett{Greift andere an:} & \fett{Hilft anderen:} \\
  \hline
  \fett{1}                       & sachlich                & sachlich \\
                                   \cline{2-3}
                                 & unsachlich              & unsachlich \\
  \hline
  \fett{2}                       & sachlich                & sachlich \\
                                   \cline{2-3}
                                 & unsachlich              & unsachlich \\
  \hline
  \fett{3}                       & sachlich                & sachlich \\
                                   \cline{2-3}
                                 & unsachlich              & unsachlich \\
  \hline
  \fett{4}                       & sachlich                & sachlich \\
                                   \cline{2-3}
                                 & unsachlich              & unsachlich \\
  \hline \hline

  \fett{Schließt Koalitionen mit:}& \fett{Hetzt gegeneinander auf:} & \fett{Vermittelt zwischen:} \\
  \hline
  1 & 1 gegen 2 & 1 und 2 \\
  \hline
  2 & 1 gegen 3 & 1 und 3 \\
  \hline
  3 & 1 gegen 4 & 1 und 4 \\
  \hline
  4 & 2 gegen 3 & 2 und 3 \\
  \hline
    & 2 gegen 4 & 2 und 4 \\
  \cline{2-3}
    & 3 gegen 4 & 3 und 4 \\
  \hline
\end{tabular}
\renewcommand{\arraystretch}{1.0}

\subsection*{Anleitung:}
In der linken Spalte werden, von oben nach unten, Name und Rolle der beobachteten Spielerinnen eingetragen, darunter Namen und Rollen ihrer Mitspielerinnen (1, 2, 3, 4).

Jedes Mal, wenn die beobachtete Spielerin etwas sagt, wird ein Strich in das passende Kästchen gesetzt. Die Zeilen 1--4 betreffen die Spielerinnen, die die beobachtete Spielerin angreift oder denen sie hilft. Also kein Strich, wenn "`Spielerin Nr.~3"' etwas sagt, sondern wenn die beobachtete Spielerin zu Spielerin 3 etwas sagt. Entsprechend werden Striche im unteren Bereich gemacht, wenn die beobachtete Spielerin zwischen zwei anderen vermittelt, oder sie gegeneinander aufhetzt.

Trotzdem wird bei der Protokollierung eines Spiels bei bestimmten Äußerungen immer noch eine Grauzone übrig bleiben, etwa wenn es darum geht zu entscheiden, ob eine Aussage nun sachlich oder unsachlich war. In wichtigen Fällen sollten die Beobachterinnen zusätzliche Anmerkungen an den Rand schreiben.
\newpage

\label{trotzburg-rollen}
\section*{Rollenbeschreibung: Krankenpfleger}

\emph{Die kleine arme mittelalterliche Stadt Trotzburg ist zerstritten mit der großen reichen Nachbarstadt Hochberg.}

Zum Krankenpfleger von Trotzburg kommt eines Tages der Schmied und sagt: "`Draußen vor der Stadt liegt ein Kaufmann aus Hochberg. Er ist verwundet. Er hat mich angefallen, da habe ich mich gewehrt und ihn zusammengeschlagen. Wir können ihn nicht im Schnee liegen lassen. Komm und hilf mir, ihn hereinzutragen!"'

Der Krankenpfleger hat wenig Lust, einem Hochberger zu helfen. Deswegen sagt er: "`Du hast mir nicht zu sagen, was ich zu tun habe. Wenn's der Bürgermeister sagt, dann gehe ich hinaus, sonst nicht!"' Eigentlich ärgert er sich ja oft über den Bürgermeister, dass dieser ihn herumkommandiert. Aber jetzt ist es ihm ganz recht, dass er auf die Autorität des Bürgermeisters verweisen kann.

Der Schmied sagt, er sei sowohl schon beim Bürgermeister gewesen und auch beim Arzt. Beide wollen nicht wirklich etwas tun. Der Krankenpfleger bleibt trotzdem hart in seiner Position.

Der Schmied läuft weg. Nach einer Weile kommt er wieder und berichtet, der Bürgermeister habe es jetzt befohlen. Da geht der Krankenpfleger mit ihm hinaus, und sie holen den Verwundeten herein.

Der Arzt verbindet seine Wunden. Trotzdem stirbt der Kaufmann noch in der Nacht. Der Arzt sagt: "`Der war nicht mehr zu retten. Die Kälte hat ihn fertig gemacht. Wenn der Wächter gleich gesehen hätte, was los war, und uns Bescheid gegeben hätte, hätte ich ihn vielleicht durchgebracht."'

\emph{Kurze Zeit darauf kommen die Soldaten von Hochberg vor die Stadt. Sie sind in der Übermacht. Sie lassen den Trotzburgern eine Botschaft überbringen: "`Liefert uns bis in einer Stunde den Schuldigen aus, der den Kaufmann getötet hat! Sonst brennen wir die ganze Stadt nieder!"'}
\newpage

\section*{Rollenbeschreibung: Schmied}

\emph{Die kleine arme mittelalterliche Stadt Trotzburg ist zerstritten mit der großen reichen Nachbarstadt Hochberg.}

Der Schmied von Trotzburg sieht eines Tages vor der Stadt einen Kaufmann aus Hochberg vorbeikommen. Er denkt sich: "`Dem nehme ich sein Geld ab!"' Er überfällt ihn, schlägt ihn zusammen und raubt ihm Geld. Beim Anblick des Verwundeten bekommt der Schmied dann aber doch Angst und rennt in die Stadt, um Hilfe zu holen.

Zuerst geht er allerdings zum Wächter auf den Turm. Der Wächter hat alles beobachtet. Der Schmied gibt ihm deswegen die Hälfte des geraubten Geldes, damit er nichts verrät. Der Wächter verspricht zu schweigen.

Danach läuft der Schmied zum Bürgermeister und sagt zu ihm: "`Eben hat mich ein Kaufmann aus Hochberg überfallen wollen. Ich habe mich gewehrt und ihn verwundet. Jetzt liegt er draußen im Schnee."' Der Bürgermeister zählt gerade die Stadtkasse nach und sagt nur: "`Das werden wir schon wieder hinbekommen!"', tut aber nichts.

Da läuft der Schmied zum Arzt und sagt: "`Komm mit vor die Stadt hinaus und hilf dem verwundeten Kaufmann!"' Der Arzt sagt: "`Was? Zu einem Hochberger soll ich hinausgehen? Fällt mir gar nicht ein. Wenn ihr ihn hereinschafft, werde ich ihn vielleicht behandeln, sonst nicht."'

Da rennt der Schmied zum Krankenpfleger und bittet ihn: "`Trage doch mit mir den Kaufmann herein! Allein schaffe ich es nicht."' Der Krankenpfleger sagt: "`Du hast mir nicht zu sagen, was ich zu tun habe. Wenn's der Bürgermeister sagt, komme ich mit, sonst nicht."'

Der Schmied läuft wieder zum Bürgermeister, der immer noch beim Geldzählen ist. Dieser sagt schließlich: "`Meinetwegen soll er ihn hereinschaffen."'

Der Schmied läuft zum Krankenpfleger, und beide tragen den Kaufmann in die Stadt. Der Arzt verbindet seine Wunden, aber noch in derselben Nacht stirbt der Kaufmann.

Der Arzt sagt: "`Der war nicht mehr zu retten. Die Kälte hat ihn fertig gemacht. Wenn der Wächter gleich gesehen hätte, was los war, und uns Bescheid gegeben hätte, hätte ich ihn vielleicht durchgebracht."'

\emph{Kurze Zeit darauf kommen die Soldaten von Hochberg vor die Stadt. Sie sind in der Übermacht. Sie lassen den Trotzburgern eine Botschaft überbringen: "`Liefert uns bis in einer Stunde den Schuldigen aus, der den Kaufmann getötet hat! Sonst brennen wir die ganze Stadt nieder!"'}
\newpage

\section*{Rollenbeschreibung: Bürgermeister}

\emph{Die kleine arme mittelalterliche Stadt Trotzburg ist zerstritten mit der großen reichen Nachbarstadt Hochberg.}

Der Bürgermeister kann die Hochberger nicht ausstehen. Eines Tages, als er gerade die Stadtkasse nachzählt, kommt der Schmied angerannt und erzählt: "`Eben hat mich ein Kaufmann aus Hochberg überfallen wollen. Ich hab mich gewehrt und ihn verwundet. Jetzt liegt er draußen im Schnee."' Der Bürgermeister denkt sich: "`Das geschieht dem Hochberger recht!"' Und weil er die Hochberger nicht mag, bleibt er hinter seinem Geld sitzen und sagt nur: "`Das werden wir schon wieder hinbekommen!"'

Der Schmied läuft daraufhin zum Arzt, aber der will auch nichts tun. Er sagt, er werde den Verwundeten vielleicht behandeln, wenn man ihn hereinbringe. Der Schmied bittet daraufhin den Krankenpfleger, zusammen mit ihm den Kaufmann hereinzutragen. Aber der sagt: "`Wenn's der Bürgermeister sagt, komme ich mit, sonst nicht."'

Da geht der Schmied zum Bürgermeister zurück und erzählt ihm alles. Der Bürgermeister sagt: "`Na, meinetwegen soll er ihn hereinschaffen."' Sie schaffen den Kaufmann herein, und der Arzt verbindet seine Wunden, aber noch in der Nacht stirbt der Kaufmann.

Der Arzt sagt zu den anderen: "`Der war nicht mehr zu retten, die Kälte hat ihn fertig gemacht. Wenn der Wächter gesehen hätte, was los war, und uns sofort Bescheid gegeben hätte, hätte ich ihn vielleicht durchgebracht."' Der Wächter sagt, er habe von dem ganzen Vorfall nichts gesehen.

\emph{Kurze Zeit darauf kommen die Soldaten von Hochberg vor die Stadt. Sie sind in der Übermacht. Sie lassen den Trotzburgern eine Botschaft überbringen: "`Liefert uns bis in einer Stunde den Schuldigen aus, der den Kaufmann getötet hat! Sonst brennen wir die ganze Stadt nieder!"'}
\newpage

\section*{Rollenbeschreibung: Arzt}

\emph{Die kleine arme mittelalterliche Stadt Trotzburg ist zerstritten mit der großen reichen Nachbarstadt Hochberg.}

Eines Tages kommt der Schmied zum Arzt und sagt: "`Draußen vor der Stadt liegt ein Kaufmann aus Hochberg verwundet im Schnee. Komm doch raus und hilf ihm! Er hat mich überfallen wollen, ich habe mich gewehrt und ihn verwundet. Eben war ich schon beim Bürgermeister, aber der will nichts unternehmen!"'

Der Arzt denkt sich: "`Geschieht ihm recht, dem Hochberger!"' Er sagt: "`Was?! Ich soll zu einem Hochberger herausgehen in dieser Kälte? Fällt mir gar nicht ein. Bringt ihn rein, dann kann ich ihn vielleicht behandeln."'

Der Schmied läuft zum Krankenpfleger und bittet ihn, den Kaufmann mit in die Stadt zu tragen. Aber der will erst etwas unternehmen, wenn es der Bürgermeister befiehlt. Daraufhin rennt der Schmied zum Bürgermeister. Dieser befiehlt endlich, den Verwundeten hereinzuschaffen. Gemeinsam tragen der Schmied und der Krankenpfleger den Kaufmann zum Arzt.

Der Arzt sieht, dass der Kaufmann sterben wird, weil er solange im Schnee gelegen hat. Er verbindet die Wunden, aber Arznei gibt er dem Kaufmann nicht, weil er sich denkt: "`Wozu soll ich diesem Hochberger auch noch kostenlos meine teure Arznei geben?"'

In der Nacht stirbt der Kaufmann. Der Arzt sagt zu den anderen: "`Der war nicht mehr zu retten, die Kälte hat ihn fertig gemacht. Wenn der Wächter gesehen hätte, was los war, und uns sofort Bescheid gegeben hätte, hätte ich ihn vielleicht durchgebracht."'

\emph{Kurze Zeit darauf kommen die Soldaten von Hochberg vor die Stadt. Sie sind in der Übermacht. Sie lassen den Trotzburgern eine Botschaft überbringen: "`Liefert uns bis in einer Stunde den Schuldigen aus, der den Kaufmann getötet hat! Sonst brennen wir die ganze Stadt nieder!"'}

Kurz vor der Beratung, in der entschieden werden soll, wer den Hochbergern als Schuldiger ausgeliefert werden soll, kommt der Wächter zum Arzt und bezahlt eine längst fällige Rechnung.
\newpage

\section*{Rollenbeschreibung: Wächter}

\emph{Die kleine arme mittelalterliche Stadt Trotzburg ist zerstritten mit der großen reichen Nachbarstadt Hochberg.}

Der Wächter steht auf seinem Turm und beobachtet die Straße, die an der Stadt vorbeiführt. Eines Tages sieht er, wie der Schmied von Trotzburg einen Hochberger Kaufmann überfällt, niederschlägt und ausraubt.

Der Wächter meldet es aber nicht in der Stadt, weil er sich denkt: "`Was geht mich ein Hochberger an?"' Kurz darauf kommt der Schmied zu ihm auf den Turm und gibt ihm Geld, damit er in der Stadt sagt, er habe nichts gesehen. Dem Wächter ist das durchaus recht, und er verspricht, nichts zu sagen.

Der Schmied läuft weiter zum Bürgermeister und stellt die Sache so dar, als wäre er vom Kaufmann angefallen worden und hätte diesen in Notwehr verwundet. Der Bürgermeister tut nichts.

Danach läuft der Schmied weiter zum Arzt, der aber nicht hinausgehen will. Er würde den Hochberger höchstens dann behandeln, wenn jemand den Verwundeten hereinbrächte. Also rennt der Schmied zum Krankenpfleger. Dieser allerdings will nur hinausgehen, wenn es der Bürgermeister befiehlt.

Nachdem der Schmied vom Bürgermeister endlich den Befehl eingeholt hat, den Kaufmann zu holen, geht der Krankenpfleger zusammen mit dem Schmied vor die Stadt.

Aber es ist schon zu spät für den Kaufmann, der in der Nacht stirbt. Der Arzt sagt: "`Der war nicht mehr zu retten, die Kälte hat ihn fertig gemacht. Wenn der Wächter gesehen hätte, dass da einer verwundet im Schnee liegt, und uns Bescheid gegeben hätte, hätte ich ihn vielleicht durchgebracht."'

\emph{Kurze Zeit darauf kommen die Soldaten von Hochberg vor die Stadt. Sie sind in der Übermacht. Sie lassen den Trotzburgern eine Botschaft überbringen: "`Liefert uns bis in einer Stunde den Schuldigen aus, der den Kaufmann getötet hat! Sonst brennen wir die ganze Stadt nieder!"'}

Kurz vor der Beratung, wer den Hochbergern als Schuldiger ausgeliefert werden soll, kommt der Wächter zum Arzt und bezahlt eine längst fällige Rechnung.

\chapter{Scotland Yard}
\label{scotland-yard-regeln}
\section*{Idee des Spiels}
Unser \emph{Scotland Yard}-Spiel leitet sich von dem bekannten Brettspiel ab. Es werden
drei Personen, \emph{Mr.~X}, \emph{Mr.~Y} und \emph{Mr.~Z} in der Bonner Innenstadt gesucht. Da wir kein Spielbrett haben, werden die Standpunkte der drei \emph{Mr.~X/Y/Z} regelmäßig per SMS an die suchenden Detektivgruppen weitergegeben.
\section*{Ziel des Spiels}
Das Ziel für jede Gruppe ist es, jeden \emph{Mr.~X/Y/Z} einmal zu fangen. Das Spiel wird
beendet, wenn eine Gruppe alle drei Personen einmal entdeckt hat.
\section*{Spielregeln}
\subsection*{Start des Spiels}
Zum Spielbeginn wird den Detektivgruppen, die alle in der PH starten, der Standpunkt der
drei zu suchenden Personen mitgeteilt.
\subsection*{Bewegung der Teilnehmer}
\subsubsection*{Mr.~X/Y/Z}
Die drei zu suchenden Personen dürfen sich nur mit den öffentlichen
Nahverkehrsmitteln oder zu Fuß bewegen. Sie dürfen nur direkte Verbindungen
zwischen Punkten auf der Karte nehmen und sich nicht zu Fuß in der Altstadt
verkrümeln.
\subsubsection*{Detektive}
Die Detektive dürfen sich bewegen, wie sie wollen.
\subsection*{Fangen eines Mr.~X/Y/Z}
Wenn die Detektive einen \emph{Mr.~X/Y/Z} finden, sprechen sie ihn an. Er wird ihnen
dann eine Unterschrift mit Uhrzeit und Ort der Gefangennahme auf den Spielplan
geben. Danach trennt sich die Detektivgruppe wieder von der gefundenen Person und sucht weiter nach den
anderen. \emph{Mr.~X/Y/Z} spielt weiter mit, damit die andern Gruppen auch eine Chance
haben ihn zu finden. Hat eine Gruppe von allen drei \emph{Mr.~X/Y/Z} eine Unterschrift,
gibt der zuletzt Gefangene eine Nachricht an die Zentrale, so dass alle vom Ende
des Spiels informiert werden.
Es soll \emph{keine} wilden Verfolgungen über verschiedene Plätze in Bonn geben. Die \emph{Mr.~X/Y/Z} werden, sobald sie die Detektive sehen, nicht panisch lossprinten, können aber noch versuchen, so gerade mit einem Bus oder einer Bahn zu entkommen.
\subsection*{Positionsangaben}
Positionsangaben von \emph{Mr.~X/Y/Z} werden alle 15~Minuten per SMS an die
Detektivgruppen gegeben. Diese Angabe enthält den Namen der Haltestelle, an der
sich \emph{Mr.~X/Y/Z} zurzeit befindet oder zuletzt befunden hat.

\emph{Mr.~X/Y/Z} bekommen keine Angaben über die suchenden Gruppen.
\subsection*{Kommunikation zwischen den Gruppen}
Die Detektivgruppen dürfen während dem Spiel miteinander telefonieren, um sich
bei der Suche abzusprechen.
\subsection*{Spielfeld}
Alle \emph{Mr.~X/Y/Z} müssen innerhalb der vorgegebenen Karte bleiben. Sie dürfen sich
nicht in irgendwelchen Gebäuden (Bars, Kneipen, etc.) aufhalten, sondern müssen
sich in der Öffentlichkeit bewegen.
\section*{Hinweise}
Der Spielplan ist nicht ganz vollständig. Einige Haltestellen,
an denen man nicht umsteigen kann, wurden zur besseren Übersichtlichkeit
weggelassen. Auch sieht es auf diesem Plan so aus, als könne man einige Strecken
einfach durchfahren. Dies muss nicht unbedingt so sein. Um genau zu sehen, wo
welche Linie fährt, hängen an den meisten Haltestellen Netzpläne von Bonn, auf
denen die Linien farbig eingezeichnet sind.

Bei einigen Haltestellen muss man darauf achten, dass die Busse nur in einer
Richtung dort vorbeifahren.

Falls ihr euch untereinander absprechen wollt, solltet ihr eure Handynummern austauschen, bevor ihr loszieht.

\section*{Scotland-Yard-Karte für das Bonner Verkehrssystem}
\scalebox{0.9}[0.9]{\includegraphics*{karte}}

\chapter{Feedback-Fragebogen zur Orientierungseinheit}
\label{fragebogen}
\medskip
LiebeR ErstsemesterIn,
\medskip
\\
deine Meinung zu unserer Orientierungseinheit ist uns wichtig, denn sie hilft uns, die nächste OE noch besser zu gestalten.

Bitte fülle diesen Fragebogen so vollständig wie möglich aus. Sei ruhig schonungslos ehrlich~-- der Fragebogen ist anonym. Wenn dir etwas besonders gut gefallen hat, darfst du uns natürlich ebenso schonungslos loben.

Vielen Dank fürs Ausfüllen!

\smallskip
\paragraph*{Dieser Tag ist heute:} $\bigcirc$ Mittwoch \hspace{1cm} $\bigcirc$ Donnerstag \hspace{1cm} $\bigcirc$ Freitag

\paragraph*{Deine TutorInnen:} \rule{10cm}{0,3pt}

\bigskip

\begin{enumerate}
\item Was hat dir heute besonders gut gefallen?
\vspace{1,0cm}
\item Was hat dir heute nicht so gut gefallen?
\vspace{1,0cm}
\item Wie fandest du die Atmosphäre in deiner Gruppe heute?
\vspace{1,0cm}
\item Was hast du heute verstanden und behalten?
\vspace{1,0cm}
\item Was hast du verstanden, aber nicht in allen Einzelheiten behalten?
\vspace{1,0cm}
\item Was hast du heute nicht verstanden?
\vspace{1,0cm}
\item Was möchtest du uns sonst noch mitteilen? (Auf der Rückseite ist auch noch Platz.)
\end{enumerate}

\chapter{Über den Autor}
Oliver Klee wohnt und arbeitet in Bonn. Er hat seit 1999 über 100 Seminare und Workshops mit einem sehr breiten Themenspektrum gegeben, zum Beispiel:
\begin{itemize}
  \item Rhetorik
  \item Kommunikation
  \item Moderation, Gruppenleitung, Train-the-Trainer
  \item Teamtrainings
  \item Zeitmanagement
  \item Entspannungsmassage
  \item Softwarearchitektur
\end{itemize}

Ihr könnt den Autor auch für Workshops zu solchen Themen engagieren.

Der Autor geht auch selbst gerne auf (gute) Seminare und ist dabei für das gelegentliche Auflockerungsspiel fast immer zu haben.

\paragraph{Website:} \url{http://www.oliverklee.de/}
\paragraph{E-Mail:} \texttt{seminare@oliverklee.de} (zum Thema Seminare) sowie \texttt{oliver@spielereader.org} (für Angelegenheiten in Sachen Spielereader)
