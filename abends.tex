\chapter{Abends bei einem Bierchen}

\section{Reise nach Kanikano}
\index{Reise nach Kanikano}
\index{Kanikano, Reise nach}
\index{Willi|see{Reise nach Kanikano}}
\index{Newscafé|see{Reise nach Kanikano}}
\index{Piratenschiff|see{Reise nach Kanikano}}
\index{Ratespaß|see{Reise nach Kanikano}}
\paragraph{Alias:} Ratespaß, Willi, Newscaf\'{e}, Piratenschiff
\paragraph{Art:} Ratespiel mit ein, zwei Eingeweihten (Eingeweiden?)
\paragraph{Ziel:} ein Muster bei der Auswahl bestimmter Gegenstände/Wörter erkennen
\paragraph{Dauer:} 15 Minuten bis mehrere Stunden (wenn's schlecht läuft)
\paragraph{Wir brauchen dazu:} einen Haufen Teilis, von denen einige das Spiel noch nicht kennen dürfen
\paragraph{So geht es:} Die Tutorin spielt eine Grenzwächterin oder Türwächterin, die die Spielerinnen nur mit bestimmten Gegenstände und Personen hineinlässt. Die Spielerinnen machen daher Vorschläge, was sie mitnehmen möchten, die die Zollbeamte dann zulässt oder ablehnt.

Wenn eine Spielerin das Muster erkannt hat, sollte sie es nicht laut sagen, sondern zuerst durch weitere Versuche zu bestätigen versuchen. Danach sollte sie den Mund halten, damit sie den anderen den Spaß nicht verdirbt. Das Spiel geht so lange, bis alle das Schema herausbekommen haben.
\paragraph{Varianten:}
	\begin{description}

		\item[Reise nach Kanikano:]
			\rotatebox{180}{
				\begin{minipage}[c]{28em}
					Die Tutorin spielt die Grenzwächterin des Landes Kanikano. Sie lässt nur Gegenstände und Personen ins Land, die in ihrem Namen \emph{kein~I} und \emph{kein~O} haben.
				\end{minipage}
			}

		\item[Willi:]
			\rotatebox{180}{
				\begin{minipage}[c]{36em}
					Willi will nur Gegenstände und Personen, in deren Namen mindestens ein Buchstabe doppelt vorkommt.
				\end{minipage}
			}

		\item[Newscafé:]
			\rotatebox{180}{
				\begin{minipage}[c]{33em}
					Ins Newscafé kommen nur Gegenstände und Personen, die den Namen einer Zeitung oder Zeitschrift tragen. Beispiel: ein Bild, deine Freundin, einen Gong, Zeit \ldots
				\end{minipage}
			}

		\item[Piratenschiff:]
			\rotatebox{180}{
				\begin{minipage}[c]{32em}
					Die Piraten auf dem Schiff (alles harte Männer und Frauen) nehmen nur mit, was mit einem Buchstaben anfängt, der in "`Piratenschiff"' enthalten ist.
				\end{minipage}
			}

		\item[Schere-Spiel/Flaschenspiel:] Siehe dort (Seite~\pageref{flaschenspiel}).
	\end{description}
\paragraph{Wann einsetzen:} Abends in geselliger Runde, wenn die Musik nicht zu laut ist.

\section{Rapunzel}
\index{Rapunzel}
\paragraph{Art:} noch ein Eingeweihten-Ratespiel
\paragraph{Ziel:} herausfinden, wem man als "`Rapunzel"' die Hand geben muss
\paragraph{Dauer:} 10--30 Minuten
\paragraph{Wir brauchen dazu:} einen Haufen Teilis, von denen einige das Spiel noch nicht kennen dürfen
\paragraph{So geht es:} Eine Spielerin ist die "`Rapunzel"' (also diejenige, die raten soll). Jemand, der das Spiel schon kennt, sagt:
  \begin{quote}
    Rapunzel, Rapunzel, hör aufs Wort!\\
    Geh nicht eher, bis ich sage: Geh!
  \end{quote}
Nach einiger Zeit schickt sie Rapunzel mit dem Wort "`Geh!"' aus dem Raum. Nach ein paar Sekunden wird Rapunzel durch die geschlossene Tür wieder hereingerufen. Sie muss dann einer Spielerin die Hand geben.

Das Ganze wiederholt sich mit dieser Rapunzel so lange, bis diese das Muster erkannt hat oder keine Lust mehr hat. Dann ist jemand anderes die Rapunzel.

\subparagraph{Lösung:}
\rotatebox{180}{
  \begin{minipage}[c]{35em}
    Rapunzel muss derjenigen die Hand geben, die nach dem obligatorischen Spruch als Erste redet. Deswegen kann Rapunzel auch erst hinausgeschickt werden, wenn jemand etwas gesagt hat.
  \end{minipage}
}
\vspace{.5em}

\paragraph{Wann einsetzen:} abends in geselliger Runde

\section{Die verrückte Professorin}
\index{Die verrückte Professorin}
\index{verrückte Professorin|see{Die verrückte Professorin}}
\index{Professorin|see{Die verrückte Professorin}}
\paragraph{Art:} total witziges Schauspiel-Ratespiel
\paragraph{Ziel:} eine Spielerin muss nur anhand von Gesten eine "`Erfindung"' erraten 
\paragraph{Dauer:} pro Runde 10--20 Minuten
\paragraph{Wir brauchen dazu:} eine Jacke, zwei Stühle sowie Sitzgelegenheiten für das Publikum
\paragraph{So geht es:} Eine Spielerin wird zur \emph{Professorin} auserkoren, die zwar sehr genial, aber auch etwas verrückt und vor allem extrem vergesslich ist. Sie hat vor kurzem eine geniale Erfindung gemacht, aber leider vergessen, was genau es war. Sie hat nur noch die Bewegung ihrer Hände, um sich wieder daran zu erinnern. 

Die Professorin verlässt zuerst den Raum. In ihrer Abwesenheit einigt sich die Gruppe auf eine "`Erfindung"', die die Professorin "`erfunden"' hat und die sich (hoffentlich) gut in Gesten darstellen lässt: Das Internet, die Kaffeemaschine, das Handy, der Laminator, der motorisierte Milchaufschäumer \ldots

Danach sucht die Gruppe noch eine zwei Teilnehmerinnen aus, die die \emph{Reporterin} sowie \emph{die Hände der Professorin} spielen.

Die "`Professorin"' wird hereingerufen, setzt sich vor der Gruppe auf den Stuhl und lässt die Arme hängen. Die "`Hände der Professorin"' kniet sich hinter die Professorin und steckt ihre Arme unter den Achseln der Professorin durch. Dann zieht sie die Jacke umgekehrt über die Arme, so dass die echten Arme der Professorin nicht zu sehen sind und das zweite Paar Arme stattdessen gestikulieren kann.

Die "`Reporterin"' setzt sich schräg gegenüber der Professorin auf einen Stuhl und fängt mit dem Interview an. Auf die Fragen antworten "`die Hände der Professorin"' mit Gesten, die die Professorin zu verstehen zu versucht und dann ihren Teil dazu sagt. Nach dem Smalltalk am Anfang werden die Fragen der Reporterin immer konkreter, bis die Professorin schließlich erraten hat, um welche Erfindung es geht. 

Beispiel:

\emph{Reporterin:} Guten Tag, Frau Professorin. Wie geht es Ihnen heute?

\emph{Hände:} (Daumen hoch.)

\emph{Professorin:} Danke, danke, sehr gut.

\emph{Reporterin:} Ich habe schon viel von Ihrer neuen Erfindung gehört. Was ist aus Ihrer Sicht das Herausragende an Ihrer Erfindung.

\emph{Hände:} (ziehen ein sehr großes Quadrat auf)

\emph{Professorin:} Dass \ldots\ dass sie sehr groß und eckig ist. Und dass man damit \ldots

\emph{Hände:} (zeigen in die Mitte)

\emph{Professorin:} \ldots\ ähm, und dass man etwas hineintun kann.

\emph{Reporterin:} Ah, sehr interessant. Und was kann man dort hineintun?

\emph{Hände:} (Trinkbewegung)

\emph{Professorin:} Ähm, Tassen. Ja, Tassen!

\paragraph{Wann einsetzen:} Abends in geselliger Runde. Wenn die Professorin und die Hände gut sind, werden sich alle vor Lachen kringeln.


\section{Assoziationsspiel}
\index{Assoziationsspiel}
\index{Heißer Stuhl|see{Assoziationsspiel}}
\paragraph{Alias:} Heißer Stuhl
\paragraph{Art:} lustiges, nicht zu actionreiches Wortspiel
\paragraph{Ziel:} Wortassoziationen finden
\paragraph{Dauer:} 5-15 Minuten (je nach Lust und Laune)
\paragraph{Wir brauchen dazu:} Stühle für alle plus ein Extrastuhl
\paragraph{So geht es:} im Stuhlkreis stehen zwei Stühle etwas abseits als "`heiße Stühle"'. Ein Stuhl davon ist frei. Jetzt sagt die Spielerin auf dem besetzen Stuhl: "`Ich bin \ldots"' und dann einen Begriff. Dann setzt sich jemand auf den freien Stuhl und sagt einen dazu passenden Begriff. Als Nächstes setzt sich jemand auf den ersten Stuhl (die Spielerin steht vom Stuhl auf und setzt sich in die Runde) und sagt wiederum einen dazu passenden Begriff.

Beispiel:

"`Ich bin der Kopfschmerz."'\\
"`Ich bin das Aspirin."'\\
"`Ich bin die Hausapotheke."'\\
"`Ich bin das Hühneraugenpflaster."'\\
"`Ich bin der Fuß."'\\
"`Ich bin die Hand."'\\
\ldots
\paragraph{Wann einsetzen:} abends zum Warmwerden oder zwischendrin zum Auflockern



\section{Kontakt}
\index{Kontakt}
\paragraph{Art:} Wort-Ratespiel, das sehr von Allgemeinbildung profitiert.
\paragraph{Ziel:} Die Gruppe soll Buchstabe für Buchstabe ein Wort raten, dass sich eine Teilnehmerin ausgesucht hat.
\paragraph{Dauer:} 1--15~Minuten pro Runde (sehr unterschiedlich)
\paragraph{Wir brauchen dazu:} 4--15 Spielerinnen
\paragraph{So geht es:} Eine Teilnehmerin ist Wortgeberin und sucht sich ein Wort aus (am besten ein Substantiv oder ein Verb), zum Beispiel ``Korrelation'', und sagt der Gruppe den Anfangsbuchstaben ``K''.

Wenn eine Teilnehmerin eine Idee hat, welches Wort es sein könnte, fragt sie eine Frage, die mit ``Ist es …'' beginnt:
\begin{quote}
``Ist es ein Tier?'' (weil die Teilnehmerin an ``Känguru'' denkt)
\end{quote}

Wenn die Wortgeberin ein anderes Wort als das zu ratende Wort kennt, das auf die Frage passt und zu den bisher bekanntgegebenen Buchstaben passt, antwortet sie mit: ``Nein, es ist kein/keine …'':

\begin{quote}
``Nein, es ist keine Kuh.''
\end{quote}

Dabei ist es unerheblich, ob dies das Wort ist, dass die Ratende im Sinn hatte.

Danach kann die Gruppe die nächste Frage stellen.

Wenn nach einer Frage aus der Gruppe eine andere Spielerin glaubt, dasselbe Wort wie die ratende Spielerin zu kennen, kann sie ``Kontakt'' rufen. Daraufhin zählen beide Spielerinnen herunter: ``Drei … zwei … eins …'' und rufen dann beide das Wort, dass sie vermuten. Dabei gibt es drei Möglichkeiten:

\begin{itemize}
  \item Wenn während des Countdowns die Wortgeberin eine Wortidee hat, sagt sie ``Nein, es ist kein(e) …'', und sonst passiert nichts.
  \item Wenn beide Teilnehmerinnen unterschiedliche Wörter sagen, passiert auch nichts.
  \item Wenn beide Teilnehmerinnen dasselbe Wort sagen, gibt die Wortgeberin der Gruppe den nächsten Buchstaben: ``Ich gebe euch ein \emph{O}!''
  \item Wenn beide Teilnehmerinnen dasselbe Wort sagen und dies das gesuchte Wort ist, hat die Gruppe das Wort erraten, und jemand anderes ist die nächste Wortgeberin. Normalerweise ist dies die Spielerin, die ``Ist es …'' gefragt hat.
\end{itemize}

\paragraph{Wann einsetzen:} abends in geselliger Runde

