\chapter{Feedback und Auswertung}
\index{Feedback}
\index{Auswertung}

\section{Standpunkte}
Auf Seite \pageref{standpunkte} zu finden.

\section{Erwartungsposter}
\index{Erwartungsposter}
\index{Poster, Erwartungs-}
\paragraph{Art:} Kartenabfrage zu Erwartungen und Befürchtungen an eine Veranstaltung
\paragraph{Ziel:} Positive und negative Erwartungen für alle sichtbar machen (vor allem für die Tutorin)
\paragraph{Dauer:} 10--15 Minuten
\paragraph{Wir brauchen dazu:} vorbereitetes Plakat, Moderationskarten in 2 Farben, Moderationsstifte, Klebestift (oder Krepp-Klebeband)
\paragraph{So geht es:} Die Tutorin hat ein Plakat mit der Überschrift \textit{Was erwarte ich von diesem Seminar (der OE etc.)?} und zwei Spalten vorbereitet:
\begin{itemize}
\item Das Seminar wird gut wenn, \ldots
\item Das Seminar wird nicht so gut wenn, \ldots
\end{itemize}
Die Hintergrundfarben der beiden Spaltenüberschriften sollten mit den beiden Farben der Moderationskarten übereinstimmen.

Wenn das Plakat hängt, bekommen alle Teilnehmerinnen Karten und Stifte. Die Tutorin sammelt die fertigen Karten ein, mischt sie und klebt sie an (geordnet nach den beiden Farben).
\paragraph{Besondere Hinweise:} Eventuell sollte die Tutorin nach dieser Aktion das Programm anpassen, wenn abzusehen ist, dass die Teilnehmerinnen etwas überhaupt nicht mögen werden.

Am Ende der Veranstaltung kann es interessant sein, zusammen mit den Teilnehmerinnen zu schauen, welche Erwartungen sich erfüllt, nicht erfüllt oder verändert haben.
\paragraph{Wann einsetzen:} Am Anfang einer Veranstaltung

\section{Blitzlicht}
\index{Blitzlicht}
\paragraph{Art:} Feedback oder Momentaufnahme.
\paragraph{Ziel:} Die Stimmung in der Gruppe wird sichtbar.
\paragraph{Dauer:} Pro Person maximal eine Minute.
\paragraph{Wir brauchen dazu:} ---
\paragraph{So geht es:} Jede Teilnehmerin bekommt eine Minute "`Sprechzeit"'. Darin kann sie ein kurzes Statement dazu abgeben, wie sie sich momentan fühlt; ob sie zufrieden ist mit dem, was sie erlebt hat; wie die Zusammenarbeit in der Gruppe klappte usw.
\paragraph{Besondere Hinweise:} Jede kommt zu Wort, die Aussagen werden nicht diskutiert oder gewertet. Auch die Tutorinnen haben die Möglichkeit etwas zu sagen.
\paragraph{Wann einsetzen:} Wenn sich Schwierigkeiten bemerkbar machen. Oder als Feedback am Ende des Tages oder zum Abschluss eines Themenbereichs.

\section{Auswertungsgalerie}
\index{Auswertungsgalerie}
\index{Galerie, Auswertungs-}
\paragraph{Art:} Feedback und Nachbereitung
\paragraph{Ziel:} Anonymes, für alle sichtbares Feedback zu einer Veranstaltung
\paragraph{Dauer:} 15--30 Minuten
\paragraph{Wir brauchen dazu:} Mehrere Pinnwände oder Wände, vorbereitete Plakate, viele Moderationsstifte
\paragraph{So geht es:} Die Tutorin hat Plakate mit Fragen vorbereitet, zu denen sie etwas von den Teilnehmerinnen erfahren möchte. Mögliche Fragen:
\begin{itemize}
\item Die Seminarmoderation: Was hat mir gut gefallen, was hat mir nicht so gefallen?
\item Wie hat mir die Unterkunft gefallen?
\item Wie ging es mir mit der Gruppe auf diesem Seminar?
\item ein großer gemalter Koffer: Was ich von diesem Seminar an Erfahrungen und Wissen mit nach Hause nehme:
\item ein großer gemalter Mülleimer: Was ich lieber hier lassen möchte:
\item Was ich sonst noch sagen möchte:
\end{itemize}
Wenn alle Plakate aufgehangen sind und genügend Stifte in der Nähe jedes Plakats liegen, können sich die Teilnehmerinnen ans Werk machen und ihre Gedanken zu Plakat bringen.

Wenn niemand mehr etwas schreiben möchte, ist die Galerie noch einmal für alle zum Anschauen eröffnet.
\paragraph{Besondere Hinweise:} Schau als Tutorin den Teilnehmerinnen nicht beim Schreiben über die Schulter! (Sonst fühlen sie sich beobachtet, und es ist nicht mehr anonym.)
\paragraph{Varianten:} Beliebt ist auch die Variante "`Koffer und Mülleimer"'.
\paragraph{Wann einsetzen:} Am Ende der OE, eines Seminars oder einer längeren Arbeitseinheit

\section{Vier Felder}
\index{Vier Felder}
\index{Felder, vier}
\index{4 Felder}
\paragraph{Art:} Feedback und Nachbereitung
\paragraph{Ziel:} Anonymes, für alle sichtbares Feedback, das zu Widersprüchen und Kritik ermutigt
\paragraph{Dauer:} 5--10 Minuten
\paragraph{Wir brauchen dazu:} Pinnwand (oder Wand), das vorbereitete Plakat, Moderationskarten in den Farben Orange, Rot, Blau und Weiß, einen Moderationsstifte pro Teilnehmerin
\paragraph{So geht es:} Die Tutorin hat ein Plakat vorbereitet, das in vier Bereiche mit je einer Frage geteilt ist. Die kursiven Wörter stehen auf einer Moderationskarte in der entsprechenden Farbe:
\begin{itemize}
\item Ein Gedanke, der mich \emph{fasziniert}: (orange Moderationskarten)
\item Ein Gedanke, dem ich \emph{nicht} zustimme: (rote Moderationskarten)
\item Was mir \emph{klar(er)} geworden ist: (blaue Moderationskarten)
\item Was mir \emph{unklar} (geblieben) ist: (weiße Moderationskarten)
\end{itemize}

Jede Teilnehmerin bekommt einen Moderationsstift sowie pro Farbe eine Moderationskarte (also insgesamt vier Karten pro Nase). Dann füllen alle ihre Karten aus und pinnen sie an. Dabei muss nicht jede Teilnehmerin für jedes Feld etwas schreiben.

Alternativ kann die Seminarleiterin auch die Karten einsammeln und aufhängen.

\paragraph{Wann einsetzen:} Am Ende eines Tages oder des kompletten Seminars.

\section{OE-Feedback-Bögen}
\index{OE-Feedback-Bögen}
\index{Feedback-Bögen}
\paragraph{Art:} schriftliche "`Umfrage"'
\paragraph{Ziel:} Feedback zur OE
\paragraph{Dauer:} 5--10 Minuten
\paragraph{Wir brauchen dazu:} pro Person einen Feedback-Bogen (Seite \pageref{fragebogen})
\paragraph{So geht es:} Alle Teilis bekommen einen Feedback-Bogen, füllen ihm anonym aus und geben ihn der Tutorin zurück.
\paragraph{Besondere Hinweise:} Die Zettel werden bei den Gruppen-Packs dabei sein. Bitte an beiden Tagen je einen Bogen ausfüllen lassen.
\paragraph{Wann einsetzen:} Donnerstag und Freitag, je am Ende der Arbeit in Gruppen. Wenn jemand früher geht, dann lasst sie bitte auch einen Bogen ausfüllen!

\section{Energiepegel-Anzeige}
\index{Energiepegel-Anzeige}
\index{Pegel-Anzeige, Energie-}
\paragraph{Art:} alle zeigen mit der Hand ihren persönlichen Energiepegel
\paragraph{Ziel:} schnelles Feedback darüber, wie viel Energie jede aus der Gruppe im Moment hat
\paragraph{Dauer:} 2 Minuten
\paragraph{Wir brauchen dazu:} Steh- oder Sitzkreis, in dem sich alle sehen können
\paragraph{So geht es:} Die Tutorin gibt vor, welche räumliche Höhe den Maximalpegel darstelle (z.\,B.~ Gürtelhöhe, Brusthöhe, Scheitel oder so). Der Boden bedeutet \textit{keine Energie}. Dann zeigen alle gleichzeitig mit der Hand, wie viel Energie sie im Moment noch haben.
\paragraph{Wann einsetzen:} Vor längeren Arbeitseinheiten, oder wenn die Gruppe insgesamt irgendwie schlapp aussieht. Oder später am Tag vor einer nicht mehr elementar wichtigen "`Zusatz-Arbeitseinheit"'.

\section{Theater}
\index{Theater}
\index{Raumschiff Enterprise|see{Theater}}
\index{Enterprise|see{Theater}}
\index{Piratenschiff|see{Theater}}
\paragraph{Alias:} Raumschiff Enterprise, Piratenschiff
\paragraph{Art:} alle positionieren sich in einem gemalten Theatergrundriss
\paragraph{Ziel:} Tages- oder Seminar-Feedback
\paragraph{Dauer:} 15--20 Minuten
\paragraph{Wir brauchen dazu:} vorbereitetes Plakat, Eddings oder Klebepunkte (evtl. in mehreren Farben)
\paragraph{So geht es:} Auf das Plakat hat die Tutorin den Grundriss eines Theaters gemalt: Bühne, Garderobe für Gäste, Foyer, Sitzplätze, Stehplätze, Loge, Regie, Maske, Künstlerinnengarderobe, Telefonzellen, Klos, Park, Technik, Bar \ldots

Alle Teilnehmerinnen tragen sich nun ein an der Stelle, an der sie sich (im übertragenen Sinne) heute (oder während des Seminars) gesehen haben. Wer sich gar nicht entscheiden kann, darf sich auch doppelt eintragen.\footnote{Oder sie trägt sich als Schmierfleck ein~-- als eine Raum-Zeit-Anomalie gewissermaßen.}
\paragraph{Varianten:} Wenn das Feedback anonym sein soll, können die Teilis auch Klebepunkte kleben, anstatt ihre Namen einzutragen.

Für ein zeitlich genaueres Feedback kann die Tutorin auch mehrere Farben benutzen (für Vormittag, Nachmittag, Abend \ldots). Dann sollte sie aber eine Legende in das Plakat integrieren.

Eine Alternative zum Theater kann auch ein anderes Gebäude, die \emph{Enterprise} oder ein Piratenschiff sein.
\paragraph{Wann einsetzen:} Am Ende eines Tages oder einer Veranstaltung

\section{Hand-Feedback}
\index{Hand-Feedback}
\index{Feedback, Hand-}
\paragraph{Art:} die fünf Finger einer Hand entsprechen fünf Fragen
\paragraph{Ziel:} Tages- oder Seminarkritik
\paragraph{Dauer:} 10--15 Minuten
\paragraph{Wir brauchen dazu:} ein Plakat, auf das eine große Hand gemalt ist, die Finger beschriftet mit den unten stehenden Fragen
\paragraph{So geht es:} Beginnend beim Daumen, geben die Teilis nacheinander ihr Feedback zu allen Fragen auf dem Plakat:
  \begin{description}
    \item[Daumen:] Daumen hoch für \ldots
    \item[Zeigefinger:] Darauf möchte ich hinweisen \ldots
    \item[Mittelfinger:] Im Mittelpunkt stand für mich \ldots
    \item[Ringfinger (mit Ring):] Mein Schmuckstück heute/auf dem Seminar war \ldots
    \item[Kleiner Finger:] Zu kurz kam für mich \ldots
  \end{description}
\paragraph{Besondere Hinweise:} Diese Methode habe ich von Vera Derschum vom v.\,f.\,h.~gelernt.
\paragraph{Wann einsetzen:} als Tages- oder Seminarkritik

\section{Ampel-Feedback}
\index{Ampel-Feedback}
\index{Feedback, Ampel-}
\index{Ampelreflexion}
\paragraph{Alias:} Ampelreflexion
\paragraph{Art:} Zustimmung zu Aussagen per Kartenheben darstellen
\paragraph{Ziel:} schnelle, überblickartige Tageskritik
\paragraph{Dauer:} 5--10 Minuten
\paragraph{Wir brauchen dazu:} Steh- oder Stuhlkreis, pro Teili je eine rote, gelbe und grüne Moderationskarte
\paragraph{So geht es:} Jede Teilnehmerin bekommt von jeder der drei Farben je eine Moderationskarte. Dann sagt nacheinander jede eine Behauptung (zum Beispiel: "`Ich habe viel Neues gelernt."'). Alle heben daraufhin eine der drei Moderationskarten, um ihre Zustimmung oder Ablehnung zu dieser Aussage zu zeigen:
\begin{itemize}
  \item grün: ich stimme zu
  \item gelb: ich weiß nicht (oder möchte mich dazu nicht äußern)
  \item rot: ich stimme nicht zu
\end{itemize}

\paragraph{Besondere Hinweise:} Danke an Björn Krüger für diese Methode.
\paragraph{Wann einsetzen:} Zur Tageskritik. Als Seminarkritik ist die Methode zu wenig qualitativ.

\section{Sektreflexion}
\index{Sektreflexion}
\index{Ich stoße an auf \ldots}
\paragraph{Alias:} Ich stoße an auf \ldots
\paragraph{Art:} Feedback durch Trinken verdeutlichen
\paragraph{Ziel:} kurzes, lustig aufgemachtes Feedback zum Tag oder Seminar
\paragraph{Dauer:} 5--10 Minuten
\paragraph{Wir brauchen dazu:} Sitz- oder Stehkreis, ein Getränk pro Person (Sekt, O-Saft, Bier \ldots)
\paragraph{So geht es:} Wer etwas sagen möchte (entweder der Reihe nach oder wer gerade Lust hat), fängt einen Satz mit einer der beiden folgenden Floskeln an:
\begin{itemize}
  \item "`Ich stoße an auch \ldots"'
  \item "'Ich spüle meinen Ärger hinunter über \ldots"'
\end{itemize}
Nach der Aussage trinken alle einen Schluck.

\paragraph{Besondere Hinweise:} Lässt sich nur anwenden, wenn genug Vertrauen zwischen den Teilis besteht, so dass diese Kritik auch nicht-anonym äußern können.

Danke an Björn Krüger für diese Methode.
\paragraph{Wann einsetzen:} Zur Tages- oder Seminarkritik.

\section{Writer's Workshop}
\index{Writer's Workshop}
\paragraph{Art:} Die Teilnehmerinnen geben Feedback zu einem Text, während die Autorin zwar zuhört, sich aber nicht äußern darf.
\paragraph{Ziel:} Feedback an die Autorin eines Papers (oder eines anderen Textes)
\paragraph{Dauer:} bis zu einer Stunde pro Text
\paragraph{Wir brauchen dazu:} einen Stuhlkreis~-- außerdem müssen alle Teilis vorher die Texte gelesen haben, die diskutiert werden
\paragraph{So geht es:}
Der Fokus eines \emph{Writer's Workshop} liegt auf dem Paper, nicht so sehr
auf der Präsentation. Tatsächlich findet es bei einem \emph{Writer's Workshop}
gar keine Präsentation eines Papers statt, sondern alle Teilnehmerinnen müssen das Paper vorher gelesen haben.

Das Format stammt aus Schriftstellerkreisen, die es verwenden, um sich
gegenseitig Gedichte, Kurzgeschichten und Ähnliches vorzustellen. (Dabei wird
kein Vortrag gehalten, sondern eine Passage aus dem Werk vorgelesen.)

Es ist dann Anfang bis Mitte der 90er auf den Pattern-Konferenzen
(PLoP, EuroPLoP \ldots) von Richard Gabriel eingeführt worden, um
Patterns zu besprechen, und hat sich dort bis heute gehalten. Richard Gabriel hat in \cite{writers} das Format detailliert beschrieben.

\paragraph{Ablauf:}

\begin{enumerate}
\item Die Autorin liest der Runde einen ihrer Meinung nach besonders
gelungenen Absatz aus ihrem Paper vor.

\item Anschließend tritt die Autorin aus der Runde und wird zur "`Fliege an
der Wand"' ("`fly on the wall"'). Eine "`Fliege an der Wand"' ist zwar
anwesend, aber so unscheinbar, dass alle anderen Anwesenden
die "`Fliege"' ignorieren. Die "`Fliege"' hört zu und macht sich Notizen, nimmt aber
nicht an der Diskussion teil und wird auch von den anderen nicht direkt
angesprochen. (Es heißt immer "`Die Autorin hat geschrieben \ldots"', nicht "`Du
hast geschrieben \ldots"'.)

\item Die Diskussion (wie gesagt, ohne die Autorin) läuft in vier Phasen ab:

\begin{enumerate}
\item \emph{Zusammenfassung}: Zuerst fasst jemand den Inhalt des Papers zusammen.
Die anderen können dies noch kommentieren und ergänzen werden, etwa falls 
sie in einem Punkt eine unterschiedliche Auffassung vertreten.

\item \emph{Positive Bestätigung:} Die Teilnehmerinnen versuchen
darzustellen, was ihnen am Paper gefallen hat. Dies kann sowohl Inhalt
als auch Darstellung umfassen: "'neue Erkenntnis"', "`überzeugende
Argumentation"', "`gute praktische Einsetzbarkeit"', aber auch: "`gute
Illustrationen"', "`schöne Formulierungen"', "`lesbare Zeichensätze"'.

\item \emph{Verbesserungsvorschläge:} Die Teilnehmerinnen machen
Vorschläge, wie das Paper verbessert werden könnte. Kritik darf
dabei nur mit dem Hintergedanken der Verbesserung formuliert werden.
Auch hier können alle Bereiche angesprochen werden~-- Inhaltliches und
Formales: "`Der Beweis ist unklar und sollte an folgenden Stellen
näher erläutert werden."', "`Ein UML-Diagramm würde zum Verständnis
beitragen."'

\item \emph{"`Sandwich"':} In der letzten Diskussionsrunde betonen die Teilis nochmals die
positiven Merkmale des Papers: Zum einen können einigen
Teilnehmerinnen aufgrund der Verbesserungsvorschläge noch weitere positive
Merkmale klar geworden sein. Zum anderen wird so verdeutlicht, dass es
sich trotz der Kritik um einen wertvollen Beitrag handelt.
\end{enumerate}

\item Nach der Diskussion kehrt die Autorin in die Runde zurück. Sie hat
Gelegenheit, Verständnisfragen zu stellen ("`Wie war die Bemerkung zu x
gemeint?"', "`Der Hinweis y war interessant~-- welche Literatur gibt es
dazu?"'), aber sie darf an dieser Stelle keine Erklärungen oder Verteidigungen des
Papers nachreichen. (Dies kann sie in den Pausen oder bei den \emph{social
events} im kleineren Kreis nachholen.)

\item Die Teilnehmerinnen stehen auf und applaudieren der Autorin.
Anschließend erzählt jemand einen Witz oder irgendeine Anekdote, die
in keinerlei Zusammenhang zu dem Paper steht.
\end{enumerate}

Manche Punkte in dieser Vorgehensweise erscheinen vielleicht etwas
merkwürdig, insbesondere Punkt~5. Aber es zeigt sich bei \emph{Writer's Workshops},
dass alle Punkte zusammen zu einer sehr konstruktiven
Atmosphäre führen. Insgesamt erhält die Autorin wertvolles Feedback,
insbesondere, was die Verständlichkeit und Darstellung ihres Papers anbelangt,
aber auch inhaltliche Hinweise.

In jedem Schritt (insbesondere in 3.c und 5., aber auch dadurch, dass die
Autorin nicht an der Diskussion teilnimmt) hilft, zu vermeiden, dass sich
jemand angegriffen fühlt oder aber Energie in unnötige Verteidigungen steckt. Beides ist destruktiv und hilft nicht bei einem der wichtigsten Ziele wissenschaftlicher Arbeit: nämlich verständliche und
erfolgreiche Papers zu schreiben. (Es ist eine Sache, in einem Vortrag
zwanzig Teilnehmer zu überzeugen, aber es ist eine andere, ein Paper so
zu schreiben, dass es auch ohne weitere Erklärung verständlich und
überzeugend ist.)

\paragraph{Besondere Hinweise:} Vielen Dank an Pascal Costanza für diese Methode.
\paragraph{Wann einsetzen:} Um vor einer Konferenz oder auf einen Schreibworkshop den Autorinnen Feedback zu ihren Texten zu geben.

\section{Zettel auf dem Rücken}
\index{Zettel auf dem Rücken}
\index{Worte verschenken|\see{Zettel auf dem Rücken}}
\paragraph{Alias:} Worte verschenken
\paragraph{Art:} sehr nettes persönliches Feedback zwischen den Teilis
\paragraph{Ziel:} jede Teilnehmerin darf den anderen Teilnehmerinnen noch nette Nachrichten mit auf den Weg geben
\paragraph{Dauer:} 5--10 Minuten
\paragraph{Wir brauchen dazu:} pro Teili je 1~etwa A3 großes Stück Packpapier, 1 schwarzen oder blauen Moderationsstift und ein paar Streifen Moderations-Klebeband
\paragraph{So geht es:} Jede Teilnehmerin klebt einer anderen Teilnehmerin mit Klebeband ein Stück Packpapier auf den Rücken. Dann schreibt jede Teili jeder anderen Teilnehmerin, der sie noch etwas auf den Weg geben möchte, eine Nachricht auf den Zettel, den diese auf dem Rücken trägt. Es muss allerdings nicht jede Teilnehmerin allen anderen etwas aufschreiben~-- sondern nur denen, denen sie noch etwas mitteilen möchte.

Die Nachrichten sollten nach Möglichkeit positiv sein, damit niemand auf dem Nachhauseweg traurig ist.

Die Mitteilungen sind dabei pseudo-anynom: Man muss sich nicht outen, aber oft ist es trotzdem klar, von wem eine Nachricht stammt.

\paragraph{Besondere Hinweise:} Achtet darauf, dass ihr auf jeden Fall nicht-durchschreibende Moderationsstifte benutzt (also die Neuland-Stifte statt der Eddings benutzen)!
\paragraph{Wann einsetzen:} ganz am Ende des Seminars nach der "`offiziellen"' Auswertung

